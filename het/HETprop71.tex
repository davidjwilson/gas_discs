%
%                         HOBBY-EBERLY TELESCOPE
%                      OBSERVING PROPOSAL TEMPLATE
%                      McDonald Observatory Version
%
%          Version 7.1  (July 2020) Modified by A. Cochran - remove setup tables
%          Version 7.0  (May 2017) Modified by A. Cochran for new instruments
%          Version 6.0  (Jan 2007) modified by A. Cochran for web submission
%          Version 5.1  (Jan 2005) modified by A. Cochran for long-term
%          Version 5.0  (September 2004) modified by A. Cochran
%          Version 4.0 (July 2001) modified by E. Barker
%          Version 3.0  (March 2001)  modified by E. Barker
%          Version 2.0  (July 1999)  modified by N. Gaffney
%          Version 1.0  (September 1996)  written by E. Robinson
%
%  NOTE: Proposals may be printed in black & white so do NOT assume that 
%  reviewers will see color versions of your figures.  Figures should be 
%  clear without color.
%
%     Proposals can be submitted ONLY in  pdf format.  Do NOT use
%     A4 paper when producing these files as that option will cause
%     the loss of several lines of text on each page.
%
%     Experienced LaTeX users will be tempted to modify the macro
%     definitions in the style file to change margins, font styles and
%     sizes, and so on.  Resist this temptation.  The TAC will respond
%     poorly to proposals that do not fit the standard style.
%
%
\documentclass[11pt]{article}
\usepackage{epsfig,HETprop70}
\begin{document}

%
%  Enter your scientific justification below.  Note:  The text portion 
%  of the justification must fit on one page.  A second page can be added 
%  for references, figures, and tables.
%  All pages related to the science justification must conform to the 
%  standard 11pt font size and 1 inch margins on all sides.
%
\BeginSciJustification


Enter your scientific justification below.  

Note:  The text portion of the justification must fit on one page.  

A second page can be added for references, figures, and tables.

To add a page for references/figures/tables, uncomment the newpage command.

All pages related to the science justification must conform to the 
standard 11pt font size and 1 inch margins (all sides).

\EndSciJustification

%  To add a page for references, figures, and tables, uncomment the following
%  command
%
%  \newpage



%
%   Describe the observations and justify the exposure times here.  
%   Also fill out the two tables:  The "Object Table" and the 
%   "Exposure Table". 
%
\BeginDescribeObservations

This is the description of the observations.

Also fill out the three tables:  The "Object Table" and the  "Exposure Table".
For LRS2, HPF and VIRUS, there are no user-selectable setup options so do
not include a setup table.
However, for LRS2, do indicate in the text and with a note in the exposure
table whether an observation is LRS2-B or LRS2-R (blue or red). 
For and HRS, a setup table is needed.

For the most up todate Throughput and Calibration Information, PLEASE
CHECK the HET web pages.

Please use the acceptable instrument setups in the Setup Table, as they
will also be used by you in providing your Phase II documentation. 

%
%   Fill out the Object Table.  For large surveys, only enter enough objects to
%   show the range of observation parameters.
%   The "Acquisition ID Method" should be a short phrase such as:
%      Finder Chart
%      Accurate Coordinates
%      Distinctive Object
%      Blind Offset from bright star
%   Replace the sample entries with your own.  Add lines to the table as needed.
% 
\BeginObjectTable
                & \# of &    RA   &     Dec    &      & Filter or  & Acquisition Indentification \\
Object Name     & Objs. & (hh:mm) &($\pm$dd:mm)& Mag. & Wavelength & Method \\ \hline

Cyg X-1         &   1   &  21:44  & $+$38:18   & 15.5 &      V     & Accurate Coordinates \\
Cyg X-1a        &   1   &  21:44  & $+$38:18   & 15.5 &      V     & Blind Offset from bright star \\
DB WDwarfs &  94   &  00:00  &    00:00   & 16.0 &  5500\AA  & Finder Charts\\ \hline
\EndObjectTable

%                    CONFIGURATION SPECIFICATIONS
% I) The LRS2
%   For the most up to date description and calibration information Please
%    see the web site. LRS2 has fixed configurations with the only variation whether you 
%    want LRS2-B (blue) or LRS2-R (red).  Please indicate that in your text or differentiate in the
%    exposure table.
%
%    (II) VIRUS
%      VIRUS has a single setup and thus does NOT need a setup table
%    (III) HPF
%      HPF has a single setup and thus does NOT need a setup table
%
%-----------------------------------------------------------------------
% IV)  The HRS 
%       Note that this section is subject to change as HRS is re-commissioned
% 
%
% HRS_ResPower_echelle_CrossDisp_fiber_skyfibers_slicer_gascell_binning
%                           
%    ResPower = 15k, 30k, 60k, or 120k
%
%   These values set the slit width for resolving powers of 15,000, 
%     30,000, 60,000, and 120,000 respectively.
%
%   echelle = blue, central, red
%
%   These values set the echelle tilt angle for changing the blaze (central) 
%    wavelengths of the orders. 
%   Please use "central" until more information is distributed. 
%
%   fiber = 2as or 3as (as=arcsecond)
%
%   Angular diameter on the sky of the fiber to be used for feeding light 
%    from the telescope to the spectrometer. 
%
%   skyfibers = 0sky, 1sky, 2sky
%
%   The maximum number of sky fibers that can be used without order overlap 
%    depends upon the values of "CrossDisp" and "fiber".
%   The limiting wavelengths are:
%      316g and 2as: 4000+ for 0sky, 5880+ for 2sky 
%      316g and 3as: 4000+ for 0sky, 5880+ for 1sky, 7100+ for 2sky 
%      600g and 2as: 4000+ for 0sky, 4200+ for 2sky 
%      600g and 3as: 4000+ for 0sky, 4200+ for 1sky, 5100+ for 2sky 
%
%   slicer = IS0 (no image slicer, 0=zero, not the letter O)
%
%   gascell = GC0 or GC1
%
%    0 = don't use the iodine gas cell 
%    1 = use the iodine gas cell as user specified elsewhere
%
%   binning = 1x1, 1x2, 1x3, 1x4, 1x5, 2x1, 2x2, 2x3, 2x4, 2x5
%
%     The format is (column binning)x(row binning), where column 
%      binning is in the cross dispersion dimension and row binning 
%      is in the echelle dispersion dimension. 
%     Typical binnings for each "ResPower" are found on web site.
%    
%        http://rhea.as.utexas.edu/HET_hrs.html
%
%              examples:
%
%     HRS_15k_central_316g7940_2as_2sky_IS0_GC0_2x5 
%
%     HRS_30k_central_600g5271_2as_2sky_IS0_GC0_2x3 
%
%     HRS_60k_central_316g5936_3as_0sky_IS0_GC0_2x1 
%
%     HRS_120k_central_316g6948_3as_0sky_IS0_GC0_1x1 
%
%   
  
\BeginSetupTable

                &        & Resolving   \\
Object Name     & Setup  & Power \\ \hline

Cyg X-1         & lrs\_g1\_15\_OG515             & 3,600   \\ 
Cyg X-1a         & lrs\_no\_no\_R             &\--    \\ 
DB WDwarfs & HRS\_120K\_central\_316g6948\_3as\_0sky\_IS0\_GC0\_1x1 & 120,000 \\ \hline
\EndSetupTable
Note 1: Setup table only to be used with HRS (and maybe HPF). \\


%
%  The UT HET TAC has now implemented a procedure to require each program 
%    to include the average setup time for all observations.
%    The overhead should be included at 10 minutes for all HRS and MRS
%    observations and 15 minutes for all LRS observations (note that the
%    maximum overhead which may be charged is 15 minutes for HRS and MRS
%    and 20 minutes for LRS - users may use these more conservative values;
%    you could be charged less for any particular observation if it takes 
%    less time to set up).
%    The setup overhead time includes: acquisition time and CCD readout time
%    as well as other related tasks.
%
%  The Exposure time is the "CCD shutter open"
%
The UT HET TAC requires the inclusion of a minimum of 10 minutes for
overhead per visit in the
Exposure Table.  Use this value for ``req" in the overhead column
of the table.  The HET Board may allow a slightly higher maximum setup time to be charged.

\BeginExposureTable


                & S/N per & $\lambda$ & \# of & Overhead & \# of & Exposure & Total &       \\
                & resolution &   of S/N  & Visits & (\#visits x  & Objects & time/visit & Time & \\
Object Name  & element & calc. & &  req. Min) & & (Mins) & (Mins) & Notes\\ \hline
Cyg X-1         & 200 &  6563\AA &  1   &   15    &  1 & 45 & 60  &  1 \\ 
Cyg X-1a        &\-- &\--  &  20   &  300  & 1  & 1 & 320 &  2  \\ 
DB WDwarfs & 100 &  6676\AA &  3   &   45    &  1 & 1880  & 5685 &  3  \\ \hline \hline
Totals  & &  & 24 & & 96 & & 6065 & \\ 
        & &  & & & & & 101.1 & Hours \\ \hline
\EndExposureTable
Note 1:  LRS2-R spectroscopic observation. \\
Note 2:  LRS2-B spectroscopic observation with visits being separated by at 
least one week. \\
Note 3:  HRS observation, each observation should have its own line 
in the table, but for large surveys a summary line like this is ok, 
providing the individual exposure times are used in calculating the total. \\

%
%  Describe the availability of tracks
%

\BeginDescribeTracks
Here you must indicate that you have used the web tool to check that
there are an adequate number of tracks for each of your targets.
This can be indicated in simple declarative statements.  However, it is
{\bf not} sufficient to say "I checked and there are enough tracks".
For targets in limited regions of the sky (i.e. a single object
or a small number of objects), you must indicate how many tracks meet
your needs.  If your targets are distributed widely across the whole
sky, such that limited number of tracks is not a problem, indicate that
instead.

The tool for checking availability is at \\
https://het.as.utexas.edu/HET/hetweb/ProgramPrep/hetdexcalendar.html. 
You should use the number of tracks in RED from the bottom of the page.

This section is limited to 4.2 inches or about 23 lines of text.

\EndDescribeTracks

%
%   Describe and justify any special constraints on the observations
%
%  This section is limited to 4.2 inches or about 23 lines of text.
%
\BeginJustifyConstraints

  Visits must be separated by at least one week because....
  
  These objects must be observed with the Moon down to avoid contamination
  by the scattered solar spectrum.

This section is limited to 4.35 inches or about 25 lines of text.

\EndJustifyConstraints



%
%  THE HET TAC has requested that each proposer present the status of 
%  the HET data acquired under previous proposals.
%
%  The PI may attach a SINGLE page of relevant publications based on McDonald 
%  observations.  The format of this page can be setup by the PI.
%
%   If relevant, how much time have you received on the telescopes at 
%   McDonald Observatory in the past three years, and what is the status of 
%   the projects for which you obtained the data?  List papers in print 
%   or in press reporting results from your observing time.
%
\BeginPreviousTime

The HET TAC has requested that each proposer present the status of 
the HET data acquired under previous proposals.

This section is limited to a single page.
The format of this page can be set by the PI.


\EndPreviousTime

\end{document}

% NOTE THAT AS OF JULY 2020 WE ARE NOT ACCEPTING LONG TERM proposals
% unless the call for proposals says otherwise.

%  The following section is "optional" and should be used
%  only as appropriate for long-term status.
%  See the instructions below.

%  PIs who have on-going projects which would be suitable for long-term
%  status should fill out the following page.  Please move the 
%  "\end{document}" above to below this section.
%  Please **DO NOT** include this page in your uploads if you didn't need
%  to fill it out.

\BeginLongTermJustification

% The request must indicate why you need long-term status and what
% milestones you expect to achieve by the end of the 1 year period.
% You must also include details of the numbers of hours you need in the
% current and following 2 trimesters.
%
%  Be sure to click the "Long-Term" box on page 2 of the web form.
%

Proposers who have projects suitable for long-term status should include
a single page of justification here.   This page is {\bf NOT} to be
used for extending the science justification of the proposal.

The request must indicate why you need long-term status and what
milestones you expect to achieve by the end of the 1 year period.
You must also include details of the numbers of hours you need in the
current and following 2 trimesters.

\EndLongTermJustification

%\end{document}

